\documentclass{WigReport}
\usepackage{epsfig} % so you can include .eps figures
\usepackage{alltt}  % so you can include program text
\usepackage{verbatim}
\usepackage{listings}
\usepackage{color}
\definecolor{ForestGreen}{rgb}{0.133, 0.545, 0.133}
\lstset{
  language=Promela,
  breaklines=true,
  commentstyle=\color{red},
  keywordstyle=\color{ForestGreen},
  basicstyle=\ttfamily
}
\title{Transport Control Protocol Model}
\ReportNumber{1}
\author{Alex Bourgeois (260355141) \\
Alex Selesse (260374116)}
\begin{document}

\MakeTitlePage
\tableofcontents
%\listoffigures
% \listoftables
\clearpage

\section{Introduction}

TODO Brief description of the Project

\section{Implementation Details}

\subsection{Approach}

TODO The approach that you used to implement TCP Connection Management Protocol
\\

Here are a few the things that need to be modelled:
\begin{itemize}
\item Server and client start off as \verb|CLOSED|.
\item Server can enter \verb|LISTEN| before/after the client sends \verb|SYN|.
\item Server's \verb|SYN| is corrupted. Client needs to reset.
\item Once the connection is established both client and server can send data to the other.
\item With an established connection both client and server can close the connection.
\item Client and server can both send a \verb|FIN| at the same time.
\end{itemize}

We decided to have server and client as separate \textit{proctypes}. Signals are sent between these \textit{proctypes} through two channels: \verb|toServer| and \verb|toClient|.

\subsection{Connection Established / Data Transmission State}

TODO How you implemented the retransmission mechanism in the Connection Established / Data Transmission State of the TCP

Once the connection is established both \textit{proctypes} enter a \verb|do| loop. Inside the loop each \textit{proctype} listens for a signal from the other. In order to avoid deadlock the client sends a signal first. This is an arbitrary decision. However having the client always send the first signal does not correctly model the behaviour of TCP. This was addressed by having additional signals. More specifically, \verb|TIMEOUT| and \verb|IDLE|. This way the client will non-deterministically send either a \verb|SEND| where the data is either a 0 or 1, a \verb|FIN| to close the connection, or an \verb|IDLE| to simulate doing nothing.

When a \textit{proctype} signals a \verb|SEND| the other \textit{proctype} will non-deterministically chose between whether to send and \verb|ACK| or to simulate a lost packet. Upon a lost packet, in TCP the sender will retransmit once a timeout is reached. To simulate this behaviour, the \textit{proctype} receiving the data will send a \verb|TIMEOUT| signal to the sender. The sender will then retransmit the same data until it receives and \verb|ACK|.

\subsection{Additional Mechanisms}

TODO Any additional mechanisms that you added

I don't think we did.

\section{Verification with Spin}

\subsection{Immediate and LTL Assertions}

TODO List of Immediate and LTL Assertions and Descriptions. For each assertion you incorporate, it must be accompanied by a description of what you are trying to verify. E.g., what behaviour of the TCP / retransmission mechanism you are trying to check the correctness?
\\

Immediate assertions were added in multiple places to assert that there are no unwanted signals. Here is the list of immediate assertions:
\\

IMM\_1:\\
Description:\\
This assertion makes sure that when a \textit{proctype} has sent data it should receive either an \verb|ACK| or \verb|TIMEOUT| signal from the other.
\\

IMM\_2:\\
Description:\\
This assertion makes sure that when a \textit{proctype} is in the data transmission state, it only receives the following signals from the other: \verb|SEND|, \verb|FIN|, and \verb|IDLE|.
\\

IMM\_3:\\
Description:\\
This assertion makes sure that when a \textit{proctype} is in the data transmission state, it does not break out of the \verb|do| loop.
\\

IMM\_4:\\
Description:\\
This assertion makes sure that when a \textit{proctype} is in the \verb|FIN WAIT 1| state it should only receive a \verb|FIN| or \verb|ACK| signal from the other.
\\

LTL assertions were created for ensuring the behaviour as depicted in the given Finite State Machine. The never claim of each LTL assertion was copied into the TCP model and ran to ensure that the model is functioning as expected. Here is the list of LTL assertions:
\\

LTL\_1:\\
\begin{lstlisting}
#define p (cstate == CLOSED_STATE)
#define q (cstate == CLOSED_STATE || cstate == SYN_SENT_STATE)
![]<>(p -> Xq)
\end{lstlisting}
Description:\\
This assertion makes sure that the client \textit{proctype} from the \verb|CLOSED| state can only go to the \verb|SYN_SENT| state.
Never claim:\\
\begin{lstlisting}
  #define p (cstate == CLOSED_STATE)
  #define q (cstate == CLOSED_STATE || cstate == SYN_SENT_STATE)
  never  {    /* ![]<>(p -> Xq) */
  T0_init:
    if
    :: ((p)) -> goto accept_S4
    :: (1) -> goto T0_init
    fi;
  accept_S4:
    if
    :: (! ((q)) && (p)) -> goto accept_S4
    fi;
  }
\end{lstlisting}

LTL\_2:\\
\begin{lstlisting}
#define p (sstate == CLOSED_STATE)
#define q (sstate == CLOSED_STATE || sstate == LISTEN_STATE)
![]<>(p -> Xq)
\end{lstlisting}
Description:\\
This assertion makes sure that the server \textit{proctype} from the \verb|CLOSED| state can only go to the \verb|LISTEN| state.
Never claim:\\
\begin{lstlisting}
  #define p (sstate == CLOSED_STATE)
  #define q (sstate == CLOSED_STATE || sstate == LISTEN_STATE)
  never  {    /* ![]<>(p -> Xq) */
  T0_init:
    if
    :: ((p)) -> goto accept_S4
    :: (1) -> goto T0_init
    fi;
  accept_S4:
    if
    :: (! ((q)) && (p)) -> goto accept_S4
    fi;
  }
\end{lstlisting}

LTL\_3:\\
\begin{lstlisting}
#define p (cstate == SYN_SENT_STATE)
#define q (cstate == SYN_SENT_STATE || cstate == SYN_RCVD_STATE || cstate == ESTABLISHED_CONNECTION_STATE)
![]<>(p -> Xq)
\end{lstlisting}
Description:\\
This assertion makes sure that the client \textit{proctype} from the \verb|SYN SENT| state can only go to the \verb|SYN RCVD| and \verb|ESTABLISHED CONNECTION| states.
Never claim:\\
\begin{lstlisting}
  #define p (cstate == SYN_SENT_STATE)
  #define q (cstate == SYN_SENT_STATE || cstate == SYN_RCVD_STATE || cstate == ESTABLISHED_CONNECTION_STATE)
  never  {    /* ![]<>(p -> Xq) */
  T0_init:
    if
    :: ((p)) -> goto accept_S4
    :: (1) -> goto T0_init
    fi;
  accept_S4:
    if
    :: (! ((q)) && (p)) -> goto accept_S4
    fi;
  }
\end{lstlisting}


LTL\_4:\\
\begin{lstlisting}
#define p (sstate == LISTEN_STATE)
#define q (sstate == LISTEN_STATE || sstate == SYN_RCVD_STATE)
![]<>(p -> Xq)
\end{lstlisting}
Description:\\
This assertion makes sure that the server \textit{proctype} from the \verb|LISTEN| state can only go to the \verb|SYN RCVD| state.
Never claim:\\
\begin{lstlisting}
  #define p (sstate == CLOSED_STATE)
  #define q (sstate == LISTEN_STATE || sstate == SYN_RCVD_STATE)
  ![]<>(p -> Xq)
  T0_init:
    if
    :: ((p)) -> goto accept_S4
    :: (1) -> goto T0_init
    fi;
  accept_S4:
    if
    :: (! ((q)) && (p)) -> goto accept_S4
    fi;
  }
\end{lstlisting}


LTL\_5:\\
\begin{lstlisting}
#define p (cstate == SYN_RCVD_STATE)
#define q (cstate == SYN_RCVD_STATE || cstate == SYN_SENT_STATE)
![]<>(p -> Xq)
\end{lstlisting}
Description:\\
This assertion makes sure that the client \textit{proctype} from the \verb|SYN RCVD| state can only go to the \verb|SYN SENT| state.
Never claim:\\
\begin{lstlisting}
  #define p (cstate == SYN_RCVD_STATE)
  #define q (cstate == SYN_RCVD_STATE || cstate == SYN_SENT_STATE)
  ![]<>(p -> Xq)
  T0_init:
    if
    :: ((p)) -> goto accept_S4
    :: (1) -> goto T0_init
    fi;
  accept_S4:
    if
    :: (! ((q)) && (p)) -> goto accept_S4
    fi;
  }
\end{lstlisting}


LTL\_6:\\
\begin{lstlisting}
#define p (sstate == SYN_RCVD_STATE)
#define q (sstate == SYN_RCVD_STATE || sstate == LISTEN_STATE || sstate == ESTABLISHED_CONNECTION_STATE)
![]<>(p -> Xq)
\end{lstlisting}
Description:\\
This assertion makes sure that the server \textit{proctype} from the \verb|SYN RCVD| state can only go to the \verb|LISTEN| and \verb|ESTABLISHED CONNECTION| states.
Never claim:\\
\begin{lstlisting}
  #define p (sstate == SYN_RCVD_STATE)
  #define q (sstate == SYN_RCVD_STATE || sstate == LISTEN_STATE || sstate == ESTABLISHED_CONNECTION_STATE)
  ![]<>(p -> Xq)
  T0_init:
    if
    :: ((p)) -> goto accept_S4
    :: (1) -> goto T0_init
    fi;
  accept_S4:
    if
    :: (! ((q)) && (p)) -> goto accept_S4
    fi;
  }
\end{lstlisting}


LTL\_7:\\
\begin{lstlisting}
#define p (cstate == ESTABLISHED_CONNECTION_STATE)
#define q (cstate == ESTABLISHED_CONNECTION_STATE || cstate == FIN_WAIT_1_STATE || cstate == CLOSE_WAIT_STATE)
![]<>(p -> Xq)
\end{lstlisting}
Description:\\
This assertion makes sure that the client \textit{proctype} from the \verb|ESTABLISHED CONNECTION| state can only go to the \verb|FIN WAIT 1| and \verb|CLOSE WAIT| states.
Never claim:\\
\begin{lstlisting}
  #define p (cstate == ESTABLISHED_CONNECTION_STATE)
  #define q (cstate == ESTABLISHED_CONNECTION_STATE || cstate == FIN_WAIT_1_STATE || cstate == CLOSE_WAIT_STATE)
  ![]<>(p -> Xq)
  T0_init:
    if
    :: ((p)) -> goto accept_S4
    :: (1) -> goto T0_init
    fi;
  accept_S4:
    if
    :: (! ((q)) && (p)) -> goto accept_S4
    fi;
  }
\end{lstlisting}


LTL\_8:\\
\begin{lstlisting}
#define p (sstate == ESTABLISHED_CONNECTION_STATE)
#define q (sstate == ESTABLISHED_CONNECTION_STATE || sstate == FIN_WAIT_1_STATE || sstate == CLOSE_WAIT_STATE)
![]<>(p -> Xq)
\end{lstlisting}
Description:\\
This assertion makes sure that the server \textit{proctype} from the \verb|ESTABLISHED CONNECTION| state can only go to the \verb|FIN WAIT 1| and \verb|CLOSE WAIT| states.
Never claim:\\
\begin{lstlisting}
  #define p (sstate == CLOSED_STATE)
  #define q (sstate == ESTABLISHED_CONNECTION_STATE || sstate == FIN_WAIT_1_STATE || sstate == CLOSE_WAIT_STATE)
  ![]<>(p -> Xq)
  T0_init:
    if
    :: ((p)) -> goto accept_S4
    :: (1) -> goto T0_init
    fi;
  accept_S4:
    if
    :: (! ((q)) && (p)) -> goto accept_S4
    fi;
  }
\end{lstlisting}


LTL\_9:\\
\begin{lstlisting}
#define p (cstate == FIN_WAIT_1_STATE)
#define q (cstate == FIN_WAIT_1_STATE || cstate == FIN_WAIT_2_STATE || cstate == CLOSING_STATE || cstate == TIME_WAIT_STATE)
![]<>(p -> Xq)
\end{lstlisting}
Description:\\
This assertion makes sure that the client \textit{proctype} from the \verb|FIN WAIT 1| state can only go to the \verb|FIN WAIT 2|, \verb|CLOSING|, and \verb|TIME WAIT| states.
Never claim:\\
\begin{lstlisting}
  #define p (cstate == FIN_WAIT_1_STATE)
  #define q (cstate == FIN_WAIT_1_STATE || cstate == FIN_WAIT_2_STATE || cstate == CLOSING_STATE || cstate == TIME_WAIT_STATE)
  ![]<>(p -> Xq)
  T0_init:
    if
    :: ((p)) -> goto accept_S4
    :: (1) -> goto T0_init
    fi;
  accept_S4:
    if
    :: (! ((q)) && (p)) -> goto accept_S4
    fi;
  }
\end{lstlisting}


LTL\_10:\\
\begin{lstlisting}
#define p (sstate == FIN_WAIT_1_STATE)
#define q (sstate == FIN_WAIT_1_STATE || sstate == FIN_WAIT_2_STATE || sstate == CLOSING_STATE || sstate == TIME_WAIT_STATE)
![]<>(p -> Xq)
\end{lstlisting}
Description:\\
This assertion makes sure that the server \textit{proctype} from the \verb|FIN WAIT 1| state can only go to the \verb|FIN WAIT 2|, \verb|CLOSING|, and \verb|TIME WAIT| states.
Never claim:\\
\begin{lstlisting}
  #define p (sstate == FIN_WAIT_1_STATE)
  #define q (sstate == FIN_WAIT_1_STATE || sstate == FIN_WAIT_2_STATE || sstate == CLOSING_STATE || sstate == TIME_WAIT_STATE)
  ![]<>(p -> Xq)
  T0_init:
    if
    :: ((p)) -> goto accept_S4
    :: (1) -> goto T0_init
    fi;
  accept_S4:
    if
    :: (! ((q)) && (p)) -> goto accept_S4
    fi;
  }
\end{lstlisting}


LTL\_11:\\
\begin{lstlisting}
#define p (cstate == FIN_WAIT_2_STATE)
#define q (cstate == FIN_WAIT_2_STATE || cstate == TIME_WAIT_STATE)
![]<>(p -> Xq)
\end{lstlisting}
Description:\\
This assertion makes sure that the client \textit{proctype} from the \verb|FIN WAIT 2| state can only go to the \verb|TIME WAIT| state.
Never claim:\\
\begin{lstlisting}
  #define p (cstate == FIN_WAIT_2_STATE)
  #define q (cstate == FIN_WAIT_2_STATE || cstate == TIME_WAIT_STATE)
  ![]<>(p -> Xq)
  T0_init:
    if
    :: ((p)) -> goto accept_S4
    :: (1) -> goto T0_init
    fi;
  accept_S4:
    if
    :: (! ((q)) && (p)) -> goto accept_S4
    fi;
  }
\end{lstlisting}


LTL\_12:\\
\begin{lstlisting}
#define p (sstate == FIN_WAIT_2_STATE)
#define q (sstate == FIN_WAIT_2_STATE || sstate == TIME_WAIT_STATE)
![]<>(p -> Xq)
\end{lstlisting}
Description:\\
This assertion makes sure that the server \textit{proctype} from the \verb|FIN WAIT 2| state can only go to the \verb|TIME WAIT| state.
Never claim:\\
\begin{lstlisting}
  #define p (sstate == FIN_WAIT_2_STATE)
  #define q (sstate == FIN_WAIT_2_STATE || sstate == TIME_WAIT_STATE)
  ![]<>(p -> Xq)
  T0_init:
    if
    :: ((p)) -> goto accept_S4
    :: (1) -> goto T0_init
    fi;
  accept_S4:
    if
    :: (! ((q)) && (p)) -> goto accept_S4
    fi;
  }
\end{lstlisting}


LTL\_12:\\
\begin{lstlisting}
#define p (cstate == CLOSING_STATE)
#define q (cstate == CLOSING_STATE || cstate == TIME_WAIT_STATE)
![]<>(p -> Xq)
\end{lstlisting}
Description:\\
This assertion makes sure that the client \textit{proctype} from the \verb|CLOSING| state can only go to the \verb|TIME WAIT| state.
Never claim:\\
\begin{lstlisting}
  #define p (cstate == CLOSING_STATE)
  #define q (cstate == CLOSING_STATE || cstate == TIME_WAIT_STATE)
  ![]<>(p -> Xq)
  T0_init:
    if
    :: ((p)) -> goto accept_S4
    :: (1) -> goto T0_init
    fi;
  accept_S4:
    if
    :: (! ((q)) && (p)) -> goto accept_S4
    fi;
  }
\end{lstlisting}


LTL\_13:\\
\begin{lstlisting}
#define p (sstate == CLOSING_STATE)
#define q (sstate == CLOSING_STATE || sstate == TIME_WAIT_STATE)
![]<>(p -> Xq)
\end{lstlisting}
Description:\\
This assertion makes sure that the server \textit{proctype} from the \verb|CLOSING| state can only go to the \verb|TIME WAIT| state.
Never claim:\\
\begin{lstlisting}
  #define p (sstate == CLOSING_STATE)
  #define q (sstate == CLOSING_STATE || sstate == TIME_WAIT_STATE)
  ![]<>(p -> Xq)
  T0_init:
    if
    :: ((p)) -> goto accept_S4
    :: (1) -> goto T0_init
    fi;
  accept_S4:
    if
    :: (! ((q)) && (p)) -> goto accept_S4
    fi;
  }
\end{lstlisting}


LTL\_14:\\
\begin{lstlisting}
#define p (cstate == TIME_WAIT_STATE)
#define q (cstate == TIME_WAIT_STATE || cstate == CLOSED_STATE)
![]<>(p -> Xq)
\end{lstlisting}
Description:\\
This assertion makes sure that the client \textit{proctype} from the \verb|TIME WAIT| state can only go to the \verb|CLOSED| state.
Never claim:\\
\begin{lstlisting}
  #define p (cstate == TIME_WAIT_STATE)
  #define q (cstate == TIME_WAIT_STATE || cstate == CLOSED_STATE)
  ![]<>(p -> Xq)
  T0_init:
    if
    :: ((p)) -> goto accept_S4
    :: (1) -> goto T0_init
    fi;
  accept_S4:
    if
    :: (! ((q)) && (p)) -> goto accept_S4
    fi;
  }
\end{lstlisting}


LTL\_15:\\
\begin{lstlisting}
#define p (sstate == TIME_WAIT_STATE)
#define q (sstate == TIME_WAIT_STATE || sstate == CLOSED_STATE)
![]<>(p -> Xq)
\end{lstlisting}
Description:\\
This assertion makes sure that the server \textit{proctype} from the \verb|TIME WAIT| state can only go to the \verb|CLOSED| state.
Never claim:\\
\begin{lstlisting}
  #define p (sstate == TIME_WAIT_STATE)
  #define q (sstate == TIME_WAIT_STATE || sstate == CLOSED_STATE)
  ![]<>(p -> Xq)
  T0_init:
    if
    :: ((p)) -> goto accept_S4
    :: (1) -> goto T0_init
    fi;
  accept_S4:
    if
    :: (! ((q)) && (p)) -> goto accept_S4
    fi;
  }
\end{lstlisting}


LTL\_16:\\
\begin{lstlisting}
#define p (cstate == CLOSE_WAIT_STATE)
#define q (cstate == CLOSE_WAIT_STATE || cstate == LAST_ACK_STATE)
![]<>(p -> Xq)
\end{lstlisting}
Description:\\
This assertion makes sure that the client \textit{proctype} from the \verb|CLOSE WAIT| state can only go to the \verb|LAST ACK| state.
Never claim:\\
\begin{lstlisting}
  #define p (cstate == CLOSE_WAIT_STATE)
  #define q (cstate == CLOSE_WAIT_STATE || cstate == LAST_ACK_STATE)
  ![]<>(p -> Xq)
  T0_init:
    if
    :: ((p)) -> goto accept_S4
    :: (1) -> goto T0_init
    fi;
  accept_S4:
    if
    :: (! ((q)) && (p)) -> goto accept_S4
    fi;
  }
\end{lstlisting}


LTL\_17:\\
\begin{lstlisting}
#define p (sstate == CLOSE_WAIT_STATE)
#define q (sstate == CLOSE_WAIT_STATE || sstate == LAST_ACK_STATE)
![]<>(p -> Xq)
\end{lstlisting}
Description:\\
This assertion makes sure that the server \textit{proctype} from the \verb|CLOSE WAIT| state can only go to the \verb|LAST ACK| state.
Never claim:\\
\begin{lstlisting}
  #define p (sstate == CLOSE_WAIT_STATE)
  #define q (sstate == CLOSE_WAIT_STATE || sstate == LAST_ACK_STATE)
  ![]<>(p -> Xq)
  T0_init:
    if
    :: ((p)) -> goto accept_S4
    :: (1) -> goto T0_init
    fi;
  accept_S4:
    if
    :: (! ((q)) && (p)) -> goto accept_S4
    fi;
  }
\end{lstlisting}


LTL\_18:\\
\begin{lstlisting}
#define p (cstate == LAST_ACK_STATE)
#define q (cstate == LAST_ACK_STATE || cstate == CLOSED_STATE)
![]<>(p -> Xq)
\end{lstlisting}
Description:\\
This assertion makes sure that the client \textit{proctype} from the \verb|LAST ACK| state can only go to the \verb|CLOSED| state.
Never claim:\\
\begin{lstlisting}
  #define p (cstate == LAST_ACK_STATE)
  #define q (cstate == LAST_ACK_STATE || cstate == CLOSED_STATE)
  ![]<>(p -> Xq)
  T0_init:
    if
    :: ((p)) -> goto accept_S4
    :: (1) -> goto T0_init
    fi;
  accept_S4:
    if
    :: (! ((q)) && (p)) -> goto accept_S4
    fi;
  }
\end{lstlisting}


LTL\_19:\\
\begin{lstlisting}
#define p (sstate == LAST_ACK_STATE)
#define q (sstate == LAST_ACK_STATE || sstate == CLOSED_STATE)
![]<>(p -> Xq)
\end{lstlisting}
Description:\\
This assertion makes sure that the server \textit{proctype} from the \verb|LAST ACK| state can only go to the \verb|CLOSED| state.
Never claim:\\
\begin{lstlisting}
  #define p (sstate == LAST_ACK_STATE)
  #define q (sstate == LAST_ACK_STATE || sstate == CLOSED_STATE)
  ![]<>(p -> Xq)
  T0_init:
    if
    :: ((p)) -> goto accept_S4
    :: (1) -> goto T0_init
    fi;
  accept_S4:
    if
    :: (! ((q)) && (p)) -> goto accept_S4
    fi;
  }
\end{lstlisting}


\section{Mutation Testing}

TODO
For each mutant / fault:
∗ Label it as MUTANT\_XX (where XX is an integer)
∗ Describe the intended incorrect behaviour that your TCP protocol should generate, and indicate the location (in terms of line number) of the mutant 6
∗ Which assertion should catch the incorrect behaviour? Indicate the assertion based on the label number (E.g., “...MUTANT\_05 was caught by assertion LTL\_47...”
∗ Did the assertion (immediate or LTL) was able to detect the incorrect behaviour? If not, can you suggest another immediate or LTL assertion to add?

\section{Conclusions}

TODO Closing remarks on the implementation of your Promela model and the approach you taken to verify the TCP protocol

Promela is a language that is hard to code nicely. For projects like this it is very easy to get lost. I made use of labels to try to organize my code however that can only do so much. Had this not been the case I would have gladly made this model more general and more comprehensive. For example it would be nice to model several clients connecting to a server. In addition coming up with all these LTL assertions are long and mind-numbing. Having said that the project was interesting.

\end{document}
